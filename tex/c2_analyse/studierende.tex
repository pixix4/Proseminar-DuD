\subsection{Studierende}

Die Hauptaufgabe der Studierenden ist es die, für die Anrechnung nötigen, Dokumente zu sammeln. Dazu gehören beispielsweise eine Beschreibung der Module und die Nachweise über die erhaltenen Noten. Die Anträge können dann im Moment auf zwei Wegen erstellt werden. Entweder füllen die Studierenden die vom Prüfungsamt bereitgestellten PDF-Vorlagen von Hand oder mit einem entsprechenden PDF-Reader aus oder sie benutzen einen bereits existierenden Assistenten, welcher nach Eingabe der Daten die PDFs automatisch ausfüllt. Nach dem Ausfüllen müssen die Anträge beim Prüfungsamt persönlich abgegeben werden. Die Möglichkeit einer Einreichung per Mail oder Online gibt es momentan nicht. Nach dem Entscheidungsprozess erfahren die Studierenden das Ergebnis per Mail und finden die dazugehörigen Noten im Notenverwaltungssystem.

Durch das neue System würde sich der Prozess für die Studierenden entscheidend vereinfachen. Sie müssten dann zwar immer noch die Recherche und Sammlung der nötigen Dokumente übernehmen, können die Anträge aber komfortabler einreichen. Die Anträge könnten mit dem neuen System komplett online erstellt und eingereicht werden, ohne das die Studierenden zum Prüfungsamt gehen müssen. Nach dem Abschicken der Dokumente sollen die Studierenden eine verbindliche Bestätigung erhalten, die den Erfolg des Abschickens bescheinigt. Ähnlich wie bisher würden die Studierenden das Ergebnis der Entscheidung digital erhalten.