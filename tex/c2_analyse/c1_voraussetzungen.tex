\section{Voraussetzungen für die Anrechnung von Studien - und Prüfungsleistungen (rechtliche Anforderungen)}

Damit ein Studierender die Anrechnung von Studien- und Prüfungsleistungen beantragen kann, muss er in der Regel zunächst in einer Hochschule eingeschrieben (gewesen) sein. Hierbei wird unterschieden zwischen zwei Varianten:
\begin{enumerate}
\item Studiengangwechsel
\item Anrechnung von Studienleistungen, die außerhalb einer Hochschule erbracht wurden
\end{enumerate}


Ein Studiengangwechsel liegt vor, wenn das Studienfach, der Studienort, der angestrebte Abschluss oder die Studienform gewechselt werden \parencite{studiengangwechsel}. Hierzu ist der Antrag auf Anerkennung von Studien- und Prüfungsleistungen im Zuge eines Studiengangwechsels im Vorfeld einzureichen \parencite{antrag_anrechnung_wechsel}.
Für die Anrechnung von Studienleistungen, die außerhalb einer Hochschule erbracht werden, ist es dagegen nicht notwendig, dass der Studierende bereits in einer Hochschule eingetragen ist, da es ebenso möglich ist, diese zu Beginn eines Studiums zu beantragen. Hierfür muss der Antrag auf Anerkennung von Studien- und Prüfungsleistungen eingereicht werden \parencite{antrag_anrechnung}.

In den Prüfungsordnungen der Fakultät Informatik zu den Studiengängen Bachelor Informatik (ausgefertigt am 24.04.2016), Bachelor Medieninformatik (ausgefertigt am 24.04.2016) und Diplom Informatiker (ausgefertigt am 27.06.2017) wird im §17 die Anrechnung der Studien- und Prüfungsleistungen geregelt.
Grundsätzlich werden Leistungen angerechnet, so lang keine wesentlichen Unterschiede zwischen der erbrachten Leistung und dem anzurechnenden Modul bestehen \parencite{pruefungsordnung_ba}. Ein Spezialfall davon ist das Erbringen von Studien- und Prüfungsleistungen an einer anderen deutschen Hochschule im gleichen Studiengang. Diese Leistungen werden ohne weitere Prüfung angerechnet, da die Module identisch sind \parencite{pruefungsordnung_ba}. In Sonderfällen können Studien- und Prüfungsleistungen auch dann angerechnet werden, wenn es gravierende Unterschiede zwischen den Leistungen und den anzurechnenden Modulen gibt. Dies ist genau dann möglich, wenn die Inhalte und die Qualifikation dem Sinn und Zweck einer vorhandenen Wahlmöglichkeit innerhalb des Studiengangs entsprechen \parencite{pruefungsordnung_ba}.
Außerdem ist es möglich, insgesamt bis zu 50 \% der Studienleistungen durch Leistungen anrechnen zu lassen, die nicht an Hochschulen erbracht wurden \parencite{pruefungsordnung_ba}. Voraussetzung dafür ist, wie bei an Hochschulen erbrachten Leistungen ebenfalls, dass die Qualifikation gleichwertig ist.
Beim Anrechnen von Studien- und Prüfungsleistungen wird, sofern die Notensysteme vergleichbar sind, die Note übernommen. Sind sie es nicht, wird die Studienleistung als bestanden markiert \parencite{pruefungsordnung_ba}.

Nach Einreichung der vollständigen Unterlagen entscheidet der Prüfungsausschuss innerhalb eines Monats über die Anrechnung der erbrachten Leistungen \parencite{pruefungsordnung_ba}.
