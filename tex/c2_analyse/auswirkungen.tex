\section{Mögliche Auswirkungen auf die Stakeholder}

Veränderungen in etablierten Arbeitsprozessen können auf die Stakeholder signifikante Auswirkungen haben. Einige mögliche werden im folgenden Abschnitt analysiert.

Die Veränderungen können von den Mitarbeitern des Prüfungsamtes sowohl positiv als auch negativ aufgefasst werden. Positiv ist vor allem der Wegfall von Arbeit. Durch ein neues System könnte man das Prüfungsamt in diesem Bereich entlasten, sodass sich die Mitarbeiter stärker auf andere Aufgaben konzentrieren können. Das Prüfungsamt kann die Einführung dennoch negativ sehen. Das Verlieren von Verantwortlichkeit für bestimmte Prozesse wird häufig als schlecht empfunden. Außerdem kann die zunehmende Automatisierung Sorgen über die Zukunft der eigenen Arbeitsstelle erzeugen.

Für den Prüfungsausschuss ist diese Entwicklung deshalb vor allem positiv. Ein digitaler Prozess kann Zeit sparen, in der man sich vermehrt anderen Aufgaben zuwenden kann. Zusätzlich dazu würde für die Mitglieder vor allem der Komfort des Prozesses erhöht werden.

Im Endeffekt bietet das neue System für die Studierenden nur Vorteile. Der Komfort würde deutlich erhöht werden und es müsste sich nicht mehr Öffnungszeiten oder Ähnliches gehalten werden. Die Abgaben können problemlos mobil erledigt werden. Ansprechpartner gibt es für die Studierenden natürlich auch mit dem neuen System immer noch.

Das Hinzukommen von Verantwortlichkeiten kann negativ und positiv gesehen werden. Zum einen steigt natürlich der Arbeitsaufwand für die IT-Abteilung, zum anderen könnte dies durch zusätzliche Mitarbeiter gut ausgeglichen werden.
