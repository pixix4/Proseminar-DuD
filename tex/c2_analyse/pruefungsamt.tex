\subsection{Prüfungsamt}

Im bisherigen Prozess nehmen die Mitarbeiter des Prüfungsamts die Anträge der Studierenden entgegen und bereiten die Dokumente für die Weitergabe an den Prüfungsausschuss vor. Nach der Überprüfung der Dokumente auf Vollständigkeit werden die Anträge an den Prüfungsausschuss weitergegeben. Wenn der Ausschuss seine Entscheidung über die Anrechnung getroffen hat, nimmt das Prüfungsamt das Ergebnis entgegen und trägt etwaige Noten in das Notenverwaltungssystem ein. Den Studierenden wird anschließend Bescheid über die Entscheidung des Ausschusses gegeben.

Durch die Einführung eines neuen Systems zur Einreichung und Bearbeitung der Anträge fällt das Prüfungsamt im Idealfall aus dem gesamten Prozess heraus. Die Entgegennahme und Aufbereitung der Anträge und Dokumente geschieht dann automatisiert und ohne den Einfluss des Prüfungsamtes. Das Eintragen der Ergebnisse bleibt möglicherweise Aufgabe das Amtes. Das hängt von der Automatisierbarkeit dieses Schrittes ab (TODO: rechtliches dazu klären).