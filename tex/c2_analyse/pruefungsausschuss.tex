\subsection{Prüfungsausschuss}

Momentan nehmen die Mitglieder des Prüfungsausschusses die zentrale Rolle in der Anrechnung von Studien – und Prüfungsleistungen ein. Der Prüfungsausschuss erhält die Anträge und Dokumente der Studierenden in aufbereiteter Form vom Prüfungsamt. Danach entscheidet der Ausschuss darüber, ob die Anrechnung ganz, teilweise oder gar nicht vorgenommen wird. Das Ergebnis der Entscheidung wird dann an das Prüfungsamt zurückgegeben.

Auch im neuen System würde der Prüfungsausschuss noch immer die Entscheidung über die Anrechnung treffen. Der Ablauf wäre aber einfacher und könnte schneller vollzogen werden. Die Rolle des Ausschusses ändert sich also nicht grundlegend, Entscheidungen könnten aber vereinfacht oder beschleunigt werden.