\subsection{Studienfachberatung}

Die Studienfachberatung nimmt als Stakeholder eine spezielle Rolle ein, die nur indirekt am Anrechnungsprozess beteiligt ist. Zum einen kann sie zu Beginn als beratende Instanz zur Verfügung stehen, die den Prozess durch die Beratung des Studierenden einleitet. 
Im Verlauf der Anrechnung dient sie weiterhin als Ansprechpartner und Koordinator.

In einem neuen, digitalen System würde diese Aufgabe erhalten bleiben. Zusätzlich würde die Studienfachberatung die Rolle einer Notfallinstanz erhalten. Sind beispielsweise keine Modulbeschreibungen vorhanden oder ist der Fall eingetreten, dass der Studierende die Anforderung des Systems nicht erfüllen kann, besteht die Möglichkeit zur manuellen Einreichung der entsprechenden Nachweise. 
