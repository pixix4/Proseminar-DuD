\chapter{Anonymisierung/ Pseudonymisierung von personenbezogenen Daten im Anrechnungsprozess}

Pseudonymisierung und Anonymisierung sind Mittel um eine Identität zu verbergen. 
Diese werden oft bei der Speicherung von Daten (wie zum Beispiel bei Online Accounts, Patientenakten etc.) verwendet, um persönliche Daten des Individuums zu schützen. 

Anonymisierung entfernt die Identität und personenbezogenen Daten vollständig aus dem zu anonymisierenden Datensatz. Dies beinhaltet alle Daten, die es ermöglichen, Rückschlüsse auf die Identität der betreffenden Person zu ziehen. Dabei gehen häufig auch Daten verloren, die zwar für die Nutzung relevant sind, aber mittels dritter Datensätze die Identifikation der ursprünglichen Person erheblich vereinfachen würden, wenn sie erhalten blieben.

Pseudonymisierung dagegen erhält auch die personenbezogenen Daten im Datensatz, welche durch einen Identifikator verschleiert werden. Dieser Identifikator wird meistens neu erzeugt, kann aber auch bereits existieren, sofern er den Sicherheitsstandards des Systems genügt \parencite{pseudonym}.

Die Verschleierung gelingt durch die Ersetzung der personenbezogenen Daten mit dem Identifikator. Hierbei muss beachtet werden, dass die Zuordnung von Identifikator zu Daten entweder nicht gespeichert wird oder nur sehr schwer darauf zugegriffen werden kann, da die Kombination dieser Zuordnung und dem pseudonymisierten Datensatz die Rekonstruktion des ursprünglichen Datensatzes trivialisiert (eine sogenannte “Linking Attack”). 

Gewisse personenbezogene Daten werden oft bei Vorgängen des Prüfungsamts zur Identifikation des Antragsstellers und zur eindeutigen Zuordnung von Anträgen zu Antragsstellern benötigt. Allerdings stellt sich die Frage, ob solche Daten auch als solche gespeichert oder ob sie im Sinne des Datenschutzes zuerst Pseudonymisiert und nur für einige wenige Vorgänge tatsächlich mit dem ursprünglichen Antragssteller in Verbindung gebracht werden sollten. Dadurch würde der Prozess des Prüfungsausschusses  nur mit den relevanten Daten des Studenten arbeiten und nicht mit den privaten Daten der Person. 

In der Theorie gibt es prinzipiell zwei Szenarien, in denen Pseudonymisierung anwendbar ist. Zum einen können personenbezogene Daten pseudonymisiert werden, falls sie vom Prüfungsamt (z.B. zur Weiterverarbeitung) zwischengespeichert werden. Zum anderen wäre es angebracht, solche Daten nicht “in Reinform” an den Prüfungsausschuss zur Bearbeitung des Antrags weiterzugeben, sofern sie nicht für die Bearbeitung notwendig sind (hierzu gehören Name, Geburtsdatum etc.).

Dies machen aber die Vielzahl an Anrechnungsformularen, die in Deutschland verwendet werden, nicht möglich. Die Formulare sind offensichtlich nicht anonym gestaltet und werden auch nicht über Pseudonyme zugeordnet. Der Grund dafür ist, dass jede Universität oder Hochschule die Formulare einer Zweiten dem richtigen Studenten zuordnen können muss. Das macht die Anonymisierung oder Pseudonymisierung der Daten die der Prüfungsausschuss erhält unmöglich, da zur Identifizierung der volle Name und unter anderem das Geburtsdatum genutzt wird. Da zudem die Daten die an das Prüfungsamt geschickt werden nie gespeichert werden, entfällt auch die Frage, ob und wie man diese anonym speichern kann.
