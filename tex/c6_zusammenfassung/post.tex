\section{Bescheide per Post}

Aktuell ist es aus rechtlicher Sicht nötig Bescheide über die Entscheidung des Prüfungsausschusses zum Antrag auf Anrechnung postalisch zuzustellen. Es wäre grundsätzlich ein Ziel, auf diesen Mehraufwand verzichten zu können, um Kosten und Papier einzusparen. Der Ansatz dieser Ausarbeitung beinhaltet eine zusätzliche digitale Zustellung, um Wartezeiten zu verkürzen.
Eine ausschließlich digitale Lösung wäre aus rechtlicher Sicht sogar grundsätzlich denkbar \parencite{protokoll}. Dabei müssten jedoch ein Paar Aspekte beachtet werden um rechtssicher zu sein. Dies wäre unter anderem die Notwendigkeit einer qualifizierten Signatur. Diese geht nach dem Prinzip „Haben und Wissen“ vor. Ich brauche also einen Gegenstand, wie zum Beispiel eine Chipkarte und zusätzlich noch Wissen wie zum Beispiel den ZIH-Login um eine solche Signatur zu erstellen.

Die Notwendigkeit einer Verschlüsselung wäre in dem Ganzen sicherlich ein nur sehr kleiner Stolperstein, muss aber auch beachtet werden.

Des Weiteren ist es nötig sicher gehen zu können, dass der Betreffende den Bescheid auch tatsächlich erhalten hat. Dies wäre vermutlich in Anlehnung an bestehende Banksysteme mit einer Registrierung des Online-Abrufs möglich. Es gibt also auch hier noch einiges näher zu beleuchten.
