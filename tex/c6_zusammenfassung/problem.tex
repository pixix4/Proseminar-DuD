\section{Das Problem mit der Unterschrift}

Um wirklich Papier einsparen zu können, wäre es sinnvoll, wenn Dokumente nicht per Hand unterschrieben werden müssten. Wie bereits zuvor betrachtet, gibt es auch an sich keine rechtliche Verpflichtung dies zu tun. Das Problem, warum man die Unterschrift aber trotzdem nicht einfach unbeachtet weglassen kann, liegt darin, dass es laut dem Gespräch mit Herrn Herber und Herrn Syckor unabdingbar ist, zu jedem Zeitpunkt sicherstellen zu können, dass die Person, die den Antrag gestellt hat, auch die Person ist, die sie vorgibt zu sein \parencite{protokoll}. Dies wird aktuell durch die handschriftliche Unterschrift weitestgehend sichergestellt. Wobei auch diese natürlich nicht über jeden Zweifel erhaben ist.

Eine digitale Lösung wäre eine Anbindung an das bestehende System und eine Nutzung des ZIH-Logins, welchen jeder Studierende der TU Dresden hat. Bedingung wäre jedoch, dass man dann die eigenen hinterlegten Daten nicht nachträglich selber ändern kann. Hier müsste auch noch untersucht werden welche Möglichkeiten man einer Person bietet, die zu dem Zeitpunkt noch nicht über einen solchen Login verfügt. Eine solche Lösung wäre weitestgehend sicher und würde den nötigen Nachweis über die Identität ausreichend erbringen.

Auch ein Verfahren, das auf einer Chipkarte basiert, zum Beispiel dem elektronischen Studierendenausweis, wäre grundsätzlich möglich. Dies würde aber in der Praxis zu erheblichem Mehraufwand führen, da jeder Studierende der sich etwas anrechnen lassen wollen würde ein Lesegerät für eben diese Chipkarte bräuchte. Deshalb wäre es wohl nicht sinnvoll diesen Ansatz weiter zu verfolgen, außer man wäre in der Lage dieses Problem zu lösen.
