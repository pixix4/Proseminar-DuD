\subsection{Eigenes Formular}
In diesem Abschnitt geht es um eine Erweiterung bzw. eine Neuauflage des aktuell bestehenden Assistenten zur Einreichung von Prüfungsleistungen. Der aktuelle Assistent tut was er soll, kann aber nicht eins zu eins für unsere Zwecke verwendet werden.

\subsubsection{Startseite}
Am Anfang hat man die Möglichkeit einen neuen Antrag anzulegen bzw. einen der bereits auf dem Server gespeichert ist zu laden und weiter zu bearbeiten, falls dieser noch nicht vollständig ist. Der Assistent ist nur zugänglich für eingeloggte Nutzer, dh. die benötigten Daten wie bspw. Name und Matrikelnummer kennt der Assistent bereits.

\subsubsection{Anlegen eines neuen Antrags}
Im neuen Assistenten gibt man zuerst alle Module ein, die man anrechnen möchte. Vorerst nur Modulabkürzung wenn vorhanden und Modulname. Im nächsten Schritt werden dem Antragsteller alle eingegebenen Module angezeigt.
Hier ist es nun zwingend notwendig seine Modulnote dem Modul anzufügen. Außerdem gibt es einen “HochladeButton” über den man etwaige Leistungsnachweise oder Ähnliches dem jeweiligen Modul anfügen kann. Dabei ist die Schwierigkeit, dass (wie in 3.2.3 beschrieben) dem Studierenden teilweise nur einen Notennachweis für alle Noten oder Noten Nachweise für alle einzelnen Module vorliegen. Das muss bei den Auswahl Möglichkeiten zum Hochladen beachtet werden.
Desweiteren kommt es hier zum ersten Unterschied zwischen internen und externen Studierenden. Im Falle des internen Studierenden kennt das System die Modulbeschreibungen aller Module und man braucht keine manuell hinzufügen. Bei Studierenden außerhalb der Uni wird dem Studenten die Möglichkeit gegeben den Modulen anderer Unis eine Modulbeschreibung hinzuzufügen. Warum nur die Möglichkeit? Bspw. in China gibt es gar keine Modulbeschreibungen. Dadurch wird jedem die Möglichkeit gegeben sich Module anrechnen zu lassen, ob es ohne Modulbeschreibung überhaupt abschätzbar ist, soll hier nicht weiter diskutiert werden.

\subsubsection{Fertigstellung und Generierung des Antrags}
Die wichtigsten Angaben sind nun getan. Allerdings muss das Prüfungsamt noch andere Daten mit aufnehmen. Hier gilt es zu wissen ob man sich für seinen aktuellen Studiengang etwas anrechnen möchte oder bspw. den Studiengang wechselt. Informationen wie “bisherige Studiengänge” oder “Anzahl der Fachsemester” werden ebenfalls abgefragt.
Der Assistent nimmt zum Schluss nun alle eingegebenen Daten überprüft diese auf Korrektheit und Vollständigkeit und erstellt daraus eine PDF, wie es aktuell schon der Fall ist. Dadurch muss sich das Prüfungsamt nicht umstellen und bekommt die gleichen PDFs zum Anrechnen zugesendet.

\subsubsection{Sonstiges}
Der Assistent ist in unserem System die einzige Variante die nötigen PDFs zu erstellen. Die bisherige Variante bei der man mit Stift und Papier die Dokumente ausfüllt und diese einscannt und hochlädt, entfällt. Wir wollen auf eine möglichst einheitliche, digitale Variante setzen.
