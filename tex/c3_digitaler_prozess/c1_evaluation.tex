\section{Evaluation und Adaption bestehender Umsetzungen}

Auf dem Markt existieren bereits einige Systeme, die sich zum Ziel gesetzt haben, die Abläufe und Prozesse an Hochschulen zu digitalisieren: Die sogenannten Hochschulinformationssysteme. In unserer Betrachtung möchten wir uns auf den deutschen Markt beschränken. Hier gibt es verschiedene Angebote, wie „Campus Net“, „Trainex“ und „HISinOne“ \parencite{wiki_his}. Das Ziel dieser Systeme ist es, die komplette Verwaltung einer Hochschule in eine Digitale Form zu bringen und dabei möglichst einfach bestehende Systeme einzugliedern bzw. bereits existierende Datensätze zu übernehmen. Dies reicht von einer digitalen Kartei mit allen Studentierenden und Mitarbeitern einer Universität bis hin zur Finanzplanung. Im Folgenden möchten wir uns allerdings auf den für unsere Arbeit relevanten Teil der Einbringung von Prüfungs- und Studienleistungen konzentrieren. Eine ganzheitliche Betrachtung der Systeme würde an dieser Stelle zu weit gehen und ein System, welches sich nur auf diesen Teil der Einbringung beschränkt existiert noch nicht.

Als Beispiel Software soll nun das Hochschul Informations System HISinOne der Hochschul Informations System EG (HIS) herangezogen werden \parencite{hisde}. Dieses wird beispielsweise an der Universität Göttingen eingesetzt, wo die wichtigsten Funktionen ausführlich in einem eigenen Wiki beschrieben sind \parencite{studit}. Dort findet sich auch eine eigene Seite, die den genauen Prozess zu Einreichung von Prüfungs- und Studienleistungen in ausschließlich digitaler Form beschreibt. Der Ablauf zur Ausfüllung des Online Formulars ist in drei Schritte unterteilt:
\begin{enumerate}
\item Angabe des Moduls, welches man sich anrechnen lassen möchte.
\item Angabe des Moduls, in dem man sich die bereits erzielten Creditpoints anrechnen lassen möchte.
\item Digital Signieren und Abschicken.
\end{enumerate}

Bei der Eintragung des Moduls, dass man sich anrechnen lassen möchte wird man außerdem aufgefordert, die jeweilige Modulbeschreibung und einen Bescheid über die Modulnote hochzuladen. Optional kann man noch zusätzliche Dokumente hochladen. Außerdem besteht die Möglichkeit die Schritte 1 und 2 für mehrere Module durchzuführen und so mehrere Anrechnungen gleichzeitig zu beantragen.

Die Daten werden nach der Signatur durch den Student automatisch an die zuständige Person an der entsprechenden Fakultät weitergeleitet. Die Voraussetzung, dass die Studierenden die Einreichung bequem von zu Hause erledigen kann und keine analogen Dokumente mehr zwischen Abteilungen der Universität hin- und hergereicht werden müssen, sind hier bereits gegeben. Im Hintergrund wird jedoch weiterhin das Prüfungsamt für die Bearbeitung des Antrages benötigt. Das Hochladen benötigter Dokumente ist bereits implementiert, aber nicht allgemein genug. Hier sollten Besonderheiten, wie fehlende Modulbeschreibungen oder wenn die Noten nur als komplette Liste und nicht für jedes einzelne Modul vorliegen. Nach Eingang des Antrags erhalten die Studierenden das Formular automatisiert per Email für die eigenen Unterlagen \parencite{studit_formular}. Allerdings bestehen hier Datenschutzbedenken, wenn die Emails mit sensiblen Daten der Studierenden unverschlüsselt verschickt werden. Die Frage ob eine Verschlüsselung aktuell eingesetzt wird, wurde bis zum Zeitpunkt der Fertigstellung dieser Arbeit nicht beantwortet.

Zusammenfassend lässt sich also festellen, dass bereits ein recht umfangreiches System zur Einreichung von Prüfungsleistungen auf dem Markt existiert. Allerdings lässt sich dieses nur im Gesamtpaket mit einem sehr komplexen System installieren. Dies wäre mit großem Aufwand der Einrichtung und Etablierung eines solchen Systems in allen Hochschulbereichen verbunden verbunden. Des Weiteren ist die Vorgehensweise des vorgestellten Systems datenschutzrechtlich bedenklich und das System nicht als freie Software verfügbar. Die Gestaltung des Formulars ist allerdings ein guter Ansatz für ein eigenes System.
