\section{Bewertung des digitalen Prozesses}

Es wurden mögliche Wege der Digitalisierung des Stakeholder betreffenden Teilprozesse und die zugrunde liegenden Gesetze aufgezeigt. Dabei wurde deutlich, dass die technischen und juristischen Möglichkeiten gegeben sind, um ein erwünschtes, digitales System umzusetzen. Jedoch steht eine solche Umsetzung vor organisatorischen Herausforderungen, die ohne die Mitwirkung des gesamten Europäischen Hochschulraumes (EHEA - European Higher Education Area) wenig Nutzen einbringen wird. So ermöglichen technische Gegebenheit die Entwicklung eines Interfaces, um Module digital erkennen und Anrechnung zu lassen, es gibt jedoch keinen Branchenstandard, um Module einheitlich dokumentiert darzustellen.
Zur Lösung wäre die Entwicklung eines Standards durch die EHEA wünschenswert, um erstens Modulbeschreibungen, zweitens die Dokumentation geprüfter Module und drittens die Echtheit durch elektronische Verifikation zu vereinheitlichen.
Eine weitere Herausforderung stellt die maschinelle Anrechnung dar. Solange das Vertrauen in die Datenanalyse nicht vorhanden ist, was vor allem in Fällen der Ablehnung eines Antrags problematisch ist, kann aber eine rechnergestützte Empfehlung zur Anerkennung die Arbeit des Prüfungsausschusses erheblich erleichtern.
