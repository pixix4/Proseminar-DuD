\chapter{ Motivation und Zielsetzung}

An der Technischen Universität Dresden (TUD) geht nichts mehr ohne Internet: Um sich an der Universität einzuschreiben, muss man sich auf einem Onlineportal anmelden [1], Übungsaufgaben und Folien zu Vorlesungen stehen auf den Internetseiten der jeweiligen Module zum Download bereit und das Einschreiben für Übungen und sogar Prüfungen geschieht online. Ohne Internetzugang könnte man zudem wichtige Veranstaltungen verpassen, da über diese hauptsächlich per Email informiert wird. Man sollte also meinen, dass die Studierenden jeden das Studium betreffenden Vorgang, bis auf das Ablegen von Prüfungen, von Zuhause aus erledigen können, solange sie nur über einen funktionierenden Internetanschluss verfügen.

Doch es gibt ein Verfahren, bei dem scheinbar seit zehn Jahren die Zeit stehen geblieben ist und keine Maßnahmen in Richtung Modernisierung und Digitalisierung unternommen worden sind: das Anrechnen von Studien- und Prüfungsleistungen. An vielen Fakultäten der TUD kann man den Antrag für das Anrechnen zwar online herunterladen, muss ihn aber meist ausdrucken und händisch ausfüllen, um ihn dann persönlich oder per Post dem zuständigen Prüfungsamt zukommen zu lassen. Dabei wäre es sowohl für die Studierenden gewohnheitsbedingt einfacher, alle wichtigen Unterlagen online einzureichen, als auch für das Prüfungsamt effizienter, da alle Unterlagen schon digitalisiert sind, was das Archivieren der nötigen Daten erleichtert und der Überschaubarkeit über alle Anträge dient. Eine erste Idee für ein System zur Anrechnung wäre, schon eines der vorhandenen Onlineportale (Selma, Opal, jExam, HISQIS, etc.) zu verwenden. Jedoch nutzt jede Fakultät unterschiedliche Portale, hinzu kommen die vielen unterschiedlichen Prüfungsordnungen der Fakultäten, welche dazu führen, dass jede Fakultät ihre eigenen Anrechnungsverfahren hat, was das Einführen eines einheitlichen universitätsinternen Systems erschwert.
Zudem gilt seit dem Mai 2018 eine neue DSGVO [2], mit der sich umgehend befasst werden muss, um ein rechtskonformes Speichern und Archivieren der Anrechnungsdaten zu gewährleisten.

Darum wird sich in dieser Arbeit mit der Suche nach einem effizienten und datenschutzkonformen System zur Anrechnung von Studien- und Prüfungsleistungen beschäftigen, welches sowohl das Einreichen für die Studierenden, als auch alle folgenden Prozesse für die wichtigen Stakeholder der Anrechnung vereinfacht. Um den Rahmen der Arbeit nicht zu sprengen, wird sich zunächst jedoch nur mit der Anrechnung an der Fakultät Informatik der TUD beschäftigt. Besonders wird der Schwerpunkt dabei auf Studiengangwechsel innerhalb der Fakultät und der Universität gelegt, auf das Anrechnen von Studien- und Prüfungsleistungen von anderen Universitäten aus Deutschland und dem Ausland wird ein Ausblick gegeben.

Daher wird sich diese Arbeit zunächst mit einer Analyse des bestehenden Prozesses zur Anrechnung an der Fakultät Informatik der TUD beschäftigen und dabei sowohl auf Prüfungsordnungen als auch Stakeholder des Verfahrens eingehen. Anschließend wird die Überführung dieses Verfahrens in einen digitalen Prozess geprüft. Dabei geht es um die Digitalisierbarkeit der Teilprozesse, es werden bestehende Systeme analysiert und bewertet und ein eigenes digitales System zur Anrechnung erstellt. Zuletzt folgt die Datenschutzbetrachtung bezüglich der Pseudonymisierung, bevor schlussendlich ein Fazit mit Ausblick gezogen wird.
