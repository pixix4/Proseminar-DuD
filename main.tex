\RequirePackage[ngerman=ngerman-x-latest]{hyphsubst}
\documentclass[ngerman]{tudscrreprt}
\usepackage[T1]{fontenc}
\usepackage{selinput}
\SelectInputMappings{adieresis={ä},germandbls={ß}}
\usepackage[defaultsans]{opensans}
\usepackage{babel}
\usepackage{hyperref}
\usepackage{graphicx}
\usepackage{import}


\begin{document}
\faculty{Fakultät Informatik}
\institute{Institut für Systemarchitektur}
\chair{Professur für Datenschutz und Datensicherheit}
\date{01.07.2019}
\author{Lars Westermann, Jan Zimmermann, Alexander Römmer, Luzia Franke, Marvin Herrmann, Cornell Ziepel, Robert Ludwig, Bruno Bellmann, Adrian Gollmann, Jonas Dimitrow, Marcus Köhler, Paul Maximilian Pickhardt, Matthias Sebastian Theodor Schermuly, Moritz Pflügner, Andreas Geyer, Nico Volkens}
\title{Digitalisierung der Anrechnung von Studien- und Prüfungsleistungen}
\thesis{Seminararbeit}
\supervisor{Frau Dr.-Ing.Katrin Borcea-Pfitzmann}
\maketitle

\tableofcontents

\subimport*{tex/c1_einleitung/}{main}

\subimport*{tex/c2_analyse/}{main}


\chapter{Überführung in einen digitalen Prozess}

\section{Digitalisierbarkeit der Teilprozesse (welche Stakeholder entfallen, Prüfungsamt nötig?)}

\section{Evaluation und Adaption bestehender Umsetzungen}

\section{Fachliche Beschreibung eines eigenen Systems (evtl. Untergliederung)}

\section{Bewertung des digitalen Prozesses}

\chapter{Vergleich des analogen und digitalen Ansatzes}

\chapter{Datenschutzbetrachtung bezüglich der Pseudonymisierung}

\section{Existierende Verfahren}

\section{Einbindung in den digitalen Prozess}

\chapter{Zusammenfassung und Ausblick}

\end{document}